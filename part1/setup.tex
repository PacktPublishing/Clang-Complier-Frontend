%% -*- coding:utf-8 -*-
\chapter{\textbf{Environment setup}}
\pagestyle{fancy}
\fancyhf{}
\rhead{\thepage}
\lhead{Environment setup}

In this chapter we will discuss the basic steps to be done to setup the
environment for 
future experiments with \clang. The setup is appropriate for Unix-based systems
such as Linux and Mac OS (Darwin). In addition, the reader will get important
information on how to download, configure, and build the LLVM source code. We
will continue with a short session that shows you how to build and use the LLVM
debugger (\lldb), which will be used as the primary tool for code investigation
throughout the book. Finally, we will finish with a simple clang tool that can
check C/C++ files for compilation errors. We will use LLDB for a simple debug
session for the created tool and clang internal. 

We will cover the following topics:
\begin{itemize}
\item Technical requirements
\item Prerequisite
\item Getting to known LLVM
\item Source code compilation
\item How to create a custom clang tool
\end{itemize}

\section{Technical requirements}
Downloading and building LLVM code is very easy and does not require any paid tools.
You will require the following: 
\begin{itemize}
\item Unix based OS (Linux, Darwin)
\item Command line git  
\item Build tools: CMake and Ninja
\end{itemize}
We will use debugger as the source investigation tool. LLVM has its own debugger
- LLDB. We will build it as our first tool build from LLVM monorepo.

Any build process consists of two steps. The first one is the project
configuration and the last one is the build itself. LLVM uses CMake as build
configuration tool. It also can use wide range of different build tools such as
Unix Makefiles, Ninja and so on. It can also generate project files for popular IDEs such
as Visual Studio and XCode. We are going to use Ninja as the build tool as soon
as it provides more faster build process and the most LLVM developers use it
\citep{llvm:getstarted}   

\subsection{CMake as build configuration tool}
CMake is an open-source, cross-platform build system generator. It has been used
as the primary build system for LLVM since version 3.3, which was released in
2013. 

Before LLVM began using CMake, it used autoconf, a tool that generates a
configure script that can be used to build and install software on a wide range
of Unix-like systems. However, autoconf has several limitations, such as being
difficult to use and maintain, and having poor support for cross-platform
builds. CMake was chosen as an alternative to autoconf because it addresses
these limitations and is easier to use and maintain. 

In addition to being used as the build system for LLVM, CMake is also used for
many other software projects, including Qt \citep{enwiki:qt}, OpenCV \citep{enwiki:opencv}, Google Test \citep{enwiki:gtest},
and others.

\subsection{Ninja as build tool}

Ninja is a small build system with a focus on speed. It is designed to be used
in conjunction with a build generator, such as CMake, which generates a build
file that describes the build rules for a project. 

One of the main advantages of Ninja is its speed. It is able to execute builds
much faster than other build systems, such as Unix Makefiles, by only rebuilding
the minimum set of files necessary to complete the build. This is because it
keeps track of the dependencies between build targets and only rebuilds targets
that are out of date. 

Additionally, Ninja is simple and easy to use. It has a small and
straightforward command-line interface, and the build files it uses are simple
text files that are easy to read and understand. 

Overall, Ninja is a good choice for build systems when speed is a concern, and
when a simple and easy-to-use tool is desired.

One of the most useful Ninja option is \myshell{-j}. This option allows you to
specify the number of commands to be run in parallel. You may want to specify
the number depending on the hardware you are using. 

Our next goal is to download LLVM code and investigate the project structure. We
also need to set up the necessary utilities for the build process and establish
the environment for our future experiments with LLVM code. This will ensure that
we have the tools and dependencies in place to proceed with our work
efficiently. 

\section{Getting to know LLVM}

Let's begin by covering some foundational information about LLVM, including the
project history as well as its structure. 

\subsection{Short LLVM history}
The \clang compiler is a part of LLVM project. The project was started in 2000
by Chris Lattner and Vikram 
Adve as their project at the University of Illinois at Urbana–Champaign
\citep{LLVM:CGO04}. 

LLVM was originally designed to be a next-generation code generation
infrastructure that could be used to build optimizing compilers for many
programming languages. However, it has since evolved into a full-featured
platform that can be used to build a wide variety of tools, including debuggers,
profilers, and static analysis tools.

LLVM has been widely adopted in the software industry and is used by many
companies and organizations to build a variety of tools and applications. It is
also used in academic research and teaching, and has inspired the development of
similar projects in other fields.

The project received an additional boost when Apple hired Chris Lattner in 2005
and formed a team to work on LLVM. LLVM became an integral part of the
development tools created by Apple (XCode).  

Initially, GCC (GNU Compile Collection) was used as the C/C++ frontend for
LLVM. But that had some disadvantages, one of them was related to GNU General Public
License (GPL)
that prevented the frontend usage at some proprietary projects. Another
disadvantage was the limited support for the Objective-C language in GCC at the
time, which was important for Apple. The clang project was started by Chris
Lattner in 2006 to address the issues.

\clang was originally designed as a unified parser for the C family
of languages, including C, Objective-C, C++, and Objective-C++. This unification
was intended to simplify maintenance by using a single frontend implementation
for multiple languages, rather than maintaining multiple implementations for
each language.

The project became successful very quickly. One of the primary reasons for the
success of \clang and LLVM was their modularity. Everything in LLVM is a library,
including \clang. It opened the opportunity to create a lot of amazing tools
based on \clang and LLVM, such as clang-tidy and clangd, which will be covered
later in the book (see \fullref{ch:clangtidy} and \fullref{ch:clangd}). 

LLVM and \clang have a very clear architecture and are written in C++. That
makes possible to investigate and use it by any C++ developer. As result we can
see huge community created around LLVM and extremely fast grows of its usage.   

\subsection{OS support}
We are planning to focus on OS for personal computers here, such as Linux,
Darwin and Windows. From other side the \clang usage is not limited by
the personal computers but can also be used to compile code for mobile platforms
such as iOS and different embedded systems.

\subsubsection{Linux}
The GCC (GNU Compiler Collection) is the default set of dev tools on Linux,
especially \myshell{gcc}(\myshell{g++}) is the default C/C++ compiler. The
\clang can also be used to compile source code on Linux. Moreover it mimics to
\myshell{gcc} and supports most of its options. From the other side LLVM
support might be limited for some GNU tools, for instance GNU Emacs does
not support \lldb as debugger. But despite the fact, the Linux is the most
suitable OS for LLVM development and investigation, thus we will mainly use this
OS for future examples. 

\subsubsection{Darwin}
The \clang is considered as the main build tool for Darwin. The entire
build infrastructure is based on LLVM, and \clang is the default C/C++
compiler. The developer tools, such as the debugger (\lldb), also come from
LLVM. You can get the primary developer utilities from XCode, which are
LLVM-based. However, you may need to install additional command-line tools, such
as CMake and Ninja, either as separate packages or through package systems like
MacPorts or Homebrew. For example, you can get CMake using Homebrew as
follows  
\begin{adjustwidth}{0em}{0em}
\begin{verbatim}
$ brew install cmake
\end{verbatim}
\end{adjustwidth}
or for MacPorts:
\begin{adjustwidth}{0em}{0em}
\begin{verbatim}
$ sudo port install cmake
\end{verbatim}
\end{adjustwidth}

\subsubsection{Windows}
On Windows, \clang can be used as a command-line compiler or as part of a larger
development environment such as Visual Studio. \clang on Windows includes support
for the Microsoft Visual C++ (MSVC) ABI, so you can use \clang to compile
programs that use the Microsoft C runtime library (CRT) and the C++ Standard
Library (STL). \clang also supports many of the same language features as GCC,
so  it can be used as a drop-in replacement for GCC on Windows in many cases. 

It's worth mentioning \myshell{clang-cl}\citep{llvm:clangcl}, it is a
command-line compiler driver for \clang that is designed to be used as a drop-in
replacement for the Microsoft Visual C++ (MSVC) compiler, \myshell{cl.exe}. It
was introduced as part of the \clang compiler, and is created to be used with the
LLVM toolchain. 

Like \myshell{cl.exe}, \myshell{clang-cl} is designed to be used as part of the
build process for Windows programs, and it supports many of the same
command-line options as the MSVC compiler. It can be used to compile C, C++, and
Objective-C code on Windows, and it can also be used to link object files and
libraries to create executable programs or dynamic-link libraries (DLLs). 

\begin{quote}
The development process for Windows is different from that of Unix-like systems,
which would require additional specifics that might make the book material quite
complicated. To avoid this complexity, our primary goal is to focus on
Unix-based systems, such as Linux and Darwin, and we will omit Windows-specific
examples in the book.
\end{quote}

\subsection{LLVM/clang project structure}
The \clang source is a part of LLVM monorepo (monolithic repository). LLVM
started to use monorepo in 2019 as a part of its transition to git
\citep{llvm:llvm2git} (TBD - verify the statement). The decision was driven by
number of factors such as 
better code reuse, improved efficiency and collaboration. Thus you can find all
LLVM projects in one place. As seen in \hyperref[sec:preface]{Preface}, we will
be using LLVM version 16.x in this book. The following command will allow you to
download it:  
\begin{adjustwidth}{0em}{0em}
\begin{verbatim}
$ git clone https://github.com/llvm/llvm-project.git -b release/16.x
$ cd llvm-project
\end{verbatim}
\end{adjustwidth}

The most important parts of the llvm-project that will be used in the book are
shown in 
\cref{fig:llvm_tree}. There are
\begin{itemize}
\item \myshell{lld} - the LLVM linker tool. You may want to use it as a
  replacement for standard linker tools such as GNU \myshell{ld}. 
\item \myshell{llvm} - common libraries for LLVM project
\item \myshell{clang} - the clang driver and frontend
\item \myshell{clang-tools-extra} - there are different clang tools that will be
  covered at the second part of the book
\end{itemize}
\begin{figure}[H]
  \begin{center}
    \begin{tikzpicture}
[
    every node/.append style = {draw, anchor = west, minimum height = 0.6cm,
      minimum width = 1.5cm},
    grow via three points={one child at (0.5,-0.8) and two children at (0.5,-0.8) and (0.5,-1.6)},
    edge from parent path={(\tikzparentnode\tikzparentanchor) |- (\tikzchildnode\tikzchildanchor)}]
\node {llvm-project}
child {node {...}}
child {node {lld}}
child {node {llvm}}
child {node {...}}
child {node {clang}}
child {node {clang-tools-extra}};
child {node {...}}
\end{tikzpicture}
  \end{center}
  \caption{LLVM project tree: there are projects that are important for the
    book: lldb (\lldb debugger), llvm (some common libs are located here), clang (C/C++
    compiler frontend and driver) and clang-tools-extra (different clang tools
    such as clang-tidy and clangd)}
  \label{fig:llvm_tree}
\end{figure}

Most projects have the same structure shown in
\cref{fig:typical_llvm_project}.
\begin{figure}[H]
  \begin{center}
    \begin{tikzpicture}
[
    every node/.append style = {draw, anchor = west, minimum height = 0.6cm,
      minimum width = 1.5cm}, 
    grow via three points={one child at (0.5,-0.8) and two children at (0.5,-0.8) and (0.5,-1.6)},
    edge from parent path={(\tikzparentnode\tikzparentanchor) |- (\tikzchildnode\tikzchildanchor)}]
\node {(llvm,clang)}
child {node {...}}
child {node {include}}
child {node {lib}}
child {node {...}}
child {node {test}}
child {node {unittest}};
child {node {...}}
\end{tikzpicture}
  \end{center}
  \caption{Typical LLVM project structure: the primary source code is located at
    include and lib folders. The test folder keeps LIT end-to-end tests and
    unittest folder keeps tests written for google test framework}
  \label{fig:typical_llvm_project}
\end{figure}
LLVM projects, such as \myshell{clang} or \myshell{llvm}, typically contain two
primary folders: 
\myshell{include} and \myshell{lib}. The
\myshell{include} folder contains the project interfaces (header
files), while the \myshell{lib} folder contains the
implementation. Each LLVM project has a variety of different tests, which can be
divided into two primary groups: unit tests located in the 'unittests' folder
and implemented using the Google Test framework, and end-to-end tests
implemented using the LLVM Integrated Tester (\myindex{LIT}) framework.
You can get more info about LLVM/\clang testing in \fullref{sec:LLVMTest}.

The most important projects for us are \myshell{clang} and \myshell{clang-tools-extra}. The
\myshell{clang} folder contains the frontend and driver.
\footnote{the compiler driver is used to run different stages of compilation
(parsing, optimisation, link etc.). You can get more info about it at \fullref{sec:clang_driver}}
For instance,
the lexer implementation is located in the \myshell{clang/lib/Lex}
folder. You can also see the \myshell{clang/test} folder, which
contains end-to-end tests, and the \myshell{clang/unittest} folder,
which contains unit tests for the frontend and driver. 

Another important folder is \myshell{clang-tools-extra}. It contains
some tools based on clang Abstract Syntax Tree (\myast). There are: 
\begin{itemize}
\item \myshell{clang-tools-extra/clangd} - a language server that
  provides navigation info for IDEs such as VSCode. 
\item \myshell{clang-tools-extra/clang-tidy} - a powerful lint
  framework with several hundreds of different checks. 
\item \myshell{clang-tools-extra/clang-format} - a code formatting tool.
\end{itemize}

After obtaining the source code and setting up build tools, we are ready to
compile the LLVM source code. 

\section{Source code compilation}

We are compiling our source code in debug mode to make it suitable for future
investigations with a debugger. We are using \lldb as the debugger. We will
start with an overview of the build process and finish with a concrete example
of the build for \lldb.

\subsection{Configuration with CMake}
\label{sec:config_with_cmake}
Create a build folder where the compiler and related tools will be built
\begin{adjustwidth}{0em}{0em}
\begin{verbatim}
$ mkdir build
$ cd build
\end{verbatim}
\end{adjustwidth}
The minimal configuration command looks like
\begin{adjustwidth}{0em}{0em}
\begin{verbatim}
$ cmake -DCMAKE_BUILD_TYPE=Debug ../llvm 
\end{verbatim}
\end{adjustwidth}
The command requires the build type to be specified (e.g. \myshell{Debug} in our
case) 
as well as the primary argument that points to a folder with the build
configuration file. The configuration file is stored as \myshell{CMakeLists.txt}
and is located in the \myshell{llvm} folder, which explains the
\myshell{../llvm} argument usage. The command generate \myshell{Makefile}
located at the build folder, thus you can use simple \myshell{make} command to start
the build process.

We will use more advanced configuration commands in the book. The command is
shown in  \cref{lis:basic_config}.
\begin{figure}[H]
\begin{minted}[breaklines=true,linenos=false]{text}
cmake -G Ninja -DCMAKE_BUILD_TYPE=Debug -DCMAKE_INSTALL_PREFIX=../install -DLLVM_TARGETS_TO_BUILD="X86" -DLLVM_ENABLE_PROJECTS="lldb;clang;clang-tools-extra" -DLLVM_USE_SPLIT_DWARF=ON ../llvm
\end{minted}
\caption{CMake: basic configuration used for the build at the book}
\label{lis:basic_config}
\end{figure}

The are several LLVM/cmake options specified:
\begin{itemize}
\item \myshell{-DCMAKE\_BUILD\_TYPE=Debug} sets the build
  mode. The build with debug info will be created. There is a primary build
  configuration for \clang internals investigations
\item \myshell{-DCMAKE\_INSTALL\_PREFIX=../install} specifies the
  installation folder
\item \myshell{-DLLVM\_TARGETS\_TO\_BUILD="X86"} sets exact
  targets to be build. It will avoid build unnecessary targets
\item \myshell{-DLLVM\_ENABLE\_PROJECTS="lldb;clang;clang-tools-extra"}
  specifies LLVM projects that we care about
\item \myshell{-DLLVM\_USE\_SPLIT\_DWARF=ON} - spits debug information into
  separate files. This option saves disk space as well as memory
  consumption during the LLVM build.
\end{itemize}

We used \myshell{-DLLVM\_USE\_SPLIT\_DWARF=ON} to save some space on the disc. For
instance the clang build with the option enabled takes 20Gb\footnote{The
configuration specified at \cref{lis:basic_config} was used for the build. The
build was run as \myshell{ninja clang}} space but it takes 31Gb space with the
option disabled. Note that the option requires compiler used for clang build to
support it. You might also notice that we create the build for one specific
architecture - \myshell{"X86"}. This option also saved some space for us because
otherwise all supported architecture will be built and the required space will
increase from 20GB to 27Gb.

You can save more space if you will use dynamic libraries instead of static
ones.  \myshell{-DBUILD\_SHARED\_LIBS=ON} will build each LLVM component as shared
library. The used space will be 14Gb at the case and the overall config command
will look like this
\begin{minted}[breaklines=true,linenos=false]{text}
cmake -G Ninja -DCMAKE_BUILD_TYPE=Debug -DCMAKE_INSTALL_PREFIX=../install -DLLVM_TARGETS_TO_BUILD="X86" -DLLVM_ENABLE_PROJECTS="lldb;clang;clang-tools-extra" -DLLVM_USE_SPLIT_DWARF=ON -DBUILD_SHARED_LIBS=ON ../llvm
\end{minted}


For performance purposes on Linux you might want to use \myshell{gold} linker
instead of default one. The \myshell{gold} linker is an alternative to
the GNU Linker that was developed as part of the GNU Binary Utilities (binutils)
package. It is designed to be faster and more efficient than the GNU Linker,
especially when linking large projects. One way it achieves this is by using a
more efficient algorithm for symbol resolution and a more compact file format
for the resulting executable. It can be enabled with
\myshell{-DLLVM\_USE\_LINKER=gold} option. The result configuration command will
look like. 
\begin{minted}[breaklines=true,linenos=false]{text}
cmake -G Ninja -DCMAKE_BUILD_TYPE=Debug -DCMAKE_INSTALL_PREFIX=../install -DLLVM_TARGETS_TO_BUILD="X86" -DLLVM_ENABLE_PROJECTS="lldb;clang;clang-tools-extra" -DLLVM_USE_LINKER=gold -DLLVM_USE_SPLIT_DWARF=ON -DBUILD_SHARED_LIBS=ON ../llvm
\end{minted}

The debug build can be very slow and you may want to consider an alternative. A
good compromise between debuggability and performance is the release build with
debug information. To obtain this build, you can change the
\myshell{CMAKE\_BUILD\_TYPE} flag to \myshell{RelWithDebInfo} in your overall configuration command. The command will
then look like this: 
\begin{minted}[breaklines=true,linenos=false]{text}
cmake -G Ninja -DCMAKE_BUILD_TYPE=RelWithDebInfo _DCMAKE_INSTALL_PREFIX=../install -DLLVM_TARGETS_TO_BUILD="X86" -DLLVM_ENABLE_PROJECTS="lldb;clang;clang-tools-extra" -DLLVM_USE_SPLIT_DWARF=ON ../llvm
\end{minted}

The following table keeps the list of some popular options \citep{llvm:build}.
\begin{table}[htbp]
\resizebox{\columnwidth}{!}{\
  \begin{tabular}{|l|l|}
    \hline
    Option & Description\\
    \hline
    \myshell{CMAKE\_BUILD\_TYPE} & Specifies build configuration. \\
    & Possible values
    are \myshell{Release|Debug|RelWithDebInfo|MinSizeRel}. \\
    & The \myshell{Release}
    and \myshell{RelWithDebInfo}
    are optimized for performance, \\
    & \myshell{MinSizeRel} is optimized for size. \\ 
    \hline
    \myshell{CMAKE\_INSTALL\_PREFIX} & Installation prefix \\
    \hline
    \myshell{CMAKE\_{C,CXX}\_FLAGS} & Extra C/C++ flags be used for compilation \\
    \hline
    \myshell{CMAKE\_{C,CXX}\_COMPILER} & C/C++ compiler be used for
    compilation. \\
    & You might want to specify a non-default compiler to use some \\
    & options that are not available or not supported by the default compiler \\
    \hline
    \myshell{LLVM\_ENABLE\_PROJECTS} & The projects to be enabled. We will use \myshell{clang;clang-tools-extra} \\
    \hline
    \myshell{LLVM\_USE\_LINKER} & Specifies the linker be used. \\
    & There are several
    options that include \myshell{gold} and \myshell{lld} \\
    \hline
  \end{tabular}
}
\caption{Configuration options}
\label{tbl:config}
\end{table}


\subsection{Build}
We need to call Ninja to build the desired project. The command for \clang build
will look like 
\begin{adjustwidth}{0em}{0em}
\begin{verbatim}
$ ninja clang
\end{verbatim}
\end{adjustwidth}

You can also run unit and end-to-end tests for the compiler with
\begin{adjustwidth}{0em}{0em}
\begin{verbatim}
$ ninja check-clang
\end{verbatim}
\end{adjustwidth}
The compiler binary can be found as \myshell{bin/clang} in the \myshell{build}
folder.  

%% \begin{quote}
%% It should be noted that the build command will create a build for the compiler
%% at the folder where the command was run i.e. \myshell{build}. The executable
%% will be located at \myshell{bin/clang}. We will use the clang for
%% our experiments that is build with basic configuration specified on
%% \cref{lis:basic_config}. 
%% \end{quote}


You can also install the binaries into the folder specified with
\myshell{-DCMAKE\_INSTALL\_PREFIX} option. It can be done as follows
\begin{adjustwidth}{0em}{0em}
\begin{verbatim}
$ ninja install clang
\end{verbatim}
\end{adjustwidth}
The folder \myshell{../install} (specified as the installation folder at
\cref{lis:basic_config}) 
will have the following structure
\begin{adjustwidth}{0em}{0em}
\begin{verbatim}
$ ls ../install
bin  include  lib  libexec  share
\end{verbatim}
\end{adjustwidth}

\subsection{The LLVM debugger, its build and usage}
The LLVM debugger, \lldb, has been created with a look at
\gdb (GNU debugger). Some of its commands repeat the counterparts from
\gdb. You may ask the question: "Why do we need a new debugger if we
have a good one?" The answer can be found in the different architecture
solutions used by GCC and LLVM. LLVM uses a modular architecture, and different
parts of the compiler can be reused. For example, the \clang frontend can be
reused in the debugger, resulting in actual support for modern C/C++
features. For example, the print command in \myshell{lldb} can specify any valid
language constructions and you can use some modern C++ features with the
\myshell{lldb} print command. 

In contrast, GCC uses a monolithic architecture, and it's hard to separate the
C/C++ frontend from other parts. Therefore, \gdb has to implement
language features separately, which may take some time before modern language
features implemented in GCC become available in \gdb. 

You may find some info about \lldb build and typical usage scenario in the
following example.
We are going to create a separate folder for release build  
\begin{adjustwidth}{0em}{0em}
\begin{verbatim}
$ cd llvm-project
$ mkdir release
$ cd release
\end{verbatim}
\end{adjustwidth}
We configure our project in Release mode and specify the \myshell{lldb} and
\myshell{clang} projects only 
\begin{minted}[breaklines=true,linenos=false]{text}
cmake -G Ninja -DCMAKE_BUILD_TYPE=Release -DCMAKE_INSTALL_PREFIX=../install -DLLVM_TARGETS_TO_BUILD="X86" -DLLVM_ENABLE_PROJECTS="lldb;clang" ../llvm
\end{minted}
We are going to build both \clang and \lldb using no more than 4 concurrent
processes 
\begin{adjustwidth}{0em}{0em}
\begin{verbatim}
$ ninja clang lldb -j4
\end{verbatim}
\end{adjustwidth}
You can install the created executables with the following command
\begin{adjustwidth}{0em}{0em}
\begin{verbatim}
$ ninja install-clang install-lldb
\end{verbatim}
\end{adjustwidth}
The binary will be installed into the folder specified via
\myshell{-DCMAKE\_INSTALL\_PREFIX} config command argument. 

We will use the following simple C++ program for the example debugger session
\inputminted{c++}{./src/part1/ch1_setup/main.cpp}
The program can be compiled using the following command
\begin{adjustwidth}{0em}{0em}
\begin{verbatim}
llvm-project/install/bin/clang main.cpp -o main -g -O0
\end{verbatim}
\end{adjustwidth}
As you may notice we don't use optimization (\myshell{-O0} option) and store
debug info at the binary (\myshell{-g} option).

Typical debug session for the created executable is shown in \cref{lis:lldb_session}.
\begin{figure}
\begin{minted}{shell-session}
$ llvm-project/install/bin/lldb main
(lldb) target create "./main"
...
(lldb) b main
Breakpoint 1: where = main`main + 11 at main.cpp:2:3,...
(lldb) r
Process 1443051 launched: ...
Process 1443051 stopped
* thread #1, name = 'main', stop reason = breakpoint 1.1
    frame #0: 0x000055555555513b main`main at main.cpp:2:3
   1    int main() {
-> 2      return 0;
   3    }
(lldb) q
\end{minted}
\caption{\lldb session example}
\label{lis:lldb_session}
\end{figure}
There are several actions done:
\begin{itemize}
  \item Run the debug session with \myshell{llvm-project/install/bin/lldb
    src/main}, where \myshell{main} is the exacutable we want to debug, see
    \cref{lis:lldb_session}, line 1.
  \item We set breakpoint at the \myshell{main} function, see
    \cref{lis:lldb_session}, line 4.
  \item Run the session with \myshell{r} command, see
    \cref{lis:lldb_session}, line 6.
  \item We can see that the process is interrupted at the breakpoint, see
    \cref{lis:lldb_session}, line 8, 12.
  \item We finish the session with \myshell{q} command, see
    \cref{lis:lldb_session}, line 14.
\end{itemize}

We are going to use \lldb as one of our tools for the \clang internals
investigation. We will use the same sequence of commands that is shown in
\cref{lis:lldb_session}. You can also use another debugger, such as \gdb, that
has a similar set of commands as \lldb.

\section{Test project: syntax check with clang tool}
\label{sec:setup_test_project}
For our first test project we will create a simple clang tool that runs compiler
and checks syntax for provided source file. We will create so called out-of-tree
LLVM project i.e. the project that will use LLVM but will be located outside the
main LLVM source tree.

There are several actions to be done to create the project:
\begin{itemize}
\item The required LLVM libraries and headers have to be built and installed
\item We have to create a build configuration file for our test project
\item The source code that uses LLVM has to be created
\end{itemize}

We will start with the first step and will install clang support libraries and
headers. We will use the following configuration command for CMake:
\begin{figure}[H]
\begin{minted}[breaklines=true,linenos=false]{text}
cmake -G Ninja -DCMAKE_BUILD_TYPE=Debug -DCMAKE_INSTALL_PREFIX=../install -DLLVM_TARGETS_TO_BUILD="X86" -DLLVM_ENABLE_PROJECTS="clang" -DLLVM_USE_LINKER=gold -DLLVM_USE_SPLIT_DWARF=ON -DBUILD_SHARED_LIBS=ON ../llvm
\end{minted}
\caption{LLVM CMake configuration for simple syntax check clang tool}
\label{lis:clang_tool_prepare}
\end{figure}
As you may notice, we enabled only one project: \texttt{clang}, all other
options are standard for our debug build. The command has to be run from a
created \texttt{build} folder inside LLVM source tree, as it was  suggested at
\fullref{sec:config_with_cmake}. 

The required libraries and headers can be installed with the following command
\begin{adjustwidth}{0em}{0em}
\begin{verbatim}
$ ninja install
\end{verbatim}
\end{adjustwidth}
The installation will be done into \texttt{install} folder as it was 
specified at \mintinline{text}{CMAKE_INSTALL_PREFIX} option.

We have to create two files for our project:
\begin{itemize}
\item \texttt{CMakeLists.txt} the build configuration file
\item \texttt{TestProject.cpp} the project source code
\end{itemize}

The configuration file will accept a path to the LLVM install folder via
\mintinline{text}{LLVM_HOME} environment variable. The configuration file is the
following:
\begin{figure}[H]
  \inputminted{cmake}{./src/part1/ch1_setup/syntaxcheck/CMakeLists.txt}
  \label{lis:cmake_sytax_check}
  \caption{CMake file for simple syntax check clang tool}
\end{figure}
The most important parts of the file are
\begin{itemize}
\item Line 2: We specify the project name (syntax-check). That also be the name
  for our executable 
\item Lines 4-7: Test for \mintinline{text}{LLVM_HOME} environment variable
\item Line 10: We set a path to LLVM CMake helpers
\item Line 11: We load LLVM CMake package from the paths specified at line 10
\item Line 14: We specify our source file that should be compiled
\item Line 16: We setup an additional flag for compilation:
  \texttt{-fno-rtti}. The flag is required as soon as LLVM is built without
  RTTI. This is done in an effort to reduce code and executable size
  \citep{llvm:coding_standards}.   
\item Line 18-22 We specify the required libraries to be linked to our
  program 
\end{itemize}

The source code for our tool is the following
\inputminted{c++}{./src/part1/ch1_setup/syntaxcheck/SyntaxCheck.cpp}
The most important part of the file are
\begin{itemize}
\item Line 7-9: The majority of compiler tools have the same set of command line
  arguments. LLVM command line library \citep{llvm:commandline_library}
  provides some API to process compiler command options. We setup the library
  at line 7. We also setup additional help message at lines 8-9. 
\item Line 13-18: We parse command line arguments
\item Line 19-23: We create and run our clang tool
\item Line 22: We use \texttt{clang::SyntaxOnlyAction} frontend action that
  will run syntax and semantic check on the input file. You can get more info
  about frontend actions later at \fullref{sec:frontend_action}.
\end{itemize}

We have to specify a path to LLVM \texttt{install} folder to build our tool. As
it was mentioned above, the path has to be specified via
\mintinline{text}{LLVM_HOME} environment variable. Our configuration command
(see \cref{lis:clang_tool_prepare}) specifies the path as \texttt{install}
folder inside LLVM project source tree. Thus we can build our tool as follows
\begin{adjustwidth}{0em}{0em}
\begin{verbatim}
export LLVM_HOME=<...>/llvm-project/install
mkdir build
cd build
cmake -G Ninja ..
ninja
\end{verbatim}
\end{adjustwidth}

We can run the tool as follows
\begin{adjustwidth}{0em}{0em}
\begin{verbatim}
$ ./build/syntax-check --help
USAGE: syntax-check [options] <source0> [... <sourceN>]
...
\end{verbatim}
\end{adjustwidth}

The program will successively terminate if we run it on a valid C++ source file,
but it will produce an error message if it's run on a broken C++ file:
\begin{adjustwidth}{0em}{0em}
\begin{verbatim}
$ ./syntax-check mainbroken.cpp 
...
Running without flags.
mainbroken.cpp:2:11: error: expected ';' after return statement
  return 0
          ^
          ;
1 error generated.
mainbroken.cpp.
\end{verbatim}
\end{adjustwidth}

We can also run our tool under LLDB debugger.
% lldb build/syntax-check -- ../main.cpp
\begin{figure}[H]
\begin{minted}{shell-session}
$  llvm-project/install/bin/lldb ./syntax-check -- main.cpp 
...
(lldb) b clang::ParseAST
...
(lldb) r
...
Running without flags.
Process 608249 stopped
* thread #1, name = 'syntax-check', stop reason = breakpoint 1.1
    frame #0: ... clang::ParseAST(...) at ParseAST.cpp:116:3
   113 	
   114 	void clang::ParseAST(...) {
   115 	  // Collect global stats on Decls/Stmts ...
-> 116 	  if (PrintStats) {
   117 	    Decl::EnableStatistics();
   118 	    Stmt::EnableStatistics();
   119 	  }
(lldb) c
Process 608249 resuming
Process 608249 exited with status = 0 (0x00000000) 
(lldb)
\end{minted}
\label{lis:lldb_clang_tool}
\caption{LLDB session for clang tool test project}
\end{figure}
We run \texttt{syntax-check} as the primary binary and set \texttt{main.cpp}
source 
file as an argument for the tool (\cref{lis:lldb_clang_tool}, line 1).
We also set a breakpoint at
\mintinline{c++}{clang::ParseAST} function (\cref{lis:lldb_clang_tool}, line
3). The function is the primary entry point for source code parsing. We run
program at line 5 and continue the execution after breakpoint at line 18.

We will use the same debugging techniques later in the book when we will
investigate clang source code.


\section{Summary}

In the chapter, we covered the history of the LLVM project, obtained the source
code for LLVM, and explored its internal structure. We learned about the tools
used to build LLVM, such as CMake and Ninja. We studied the various
configuration options for building LLVM and how they can be used to optimize
resources, including disk space. We built \clang and \lldb in release mode and
used  the resulting tools to compile a basic program and run it under the
debugger. We also created a simple clang tool and run it under LLDB debugger.

The next chapter will introduce you to compiler design architecture and how it
appears in the context of \clang. We will primarily focus on the \clang
frontend, but we will also cover the important concept of the \clang driver -
the backbone that manages all stages of the compilation process, from parsing to
linking.

\section{Further reading}
\begin{itemize}
\item Getting Started with the LLVM System,
  https://llvm.org/docs/GettingStarted.html
\item Building LLVM with CMake, https://llvm.org/docs/CMake.html
\item Clang Compiler User’s Manual, https://clang.llvm.org/docs/UsersManual.html
\end{itemize}
