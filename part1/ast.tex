%% -*- coding:utf-8 -*-
\chapter{Clang AST}
The \myast is the skeleton for the Clang frontend. It is also the primary
instrument for linters and other Clang tools. The AST keeps the results of
syntax and semantic analysis and represents a tree
\footnote{The Clang AST is not
a real tree since backward edges are possible, making the graph a more suitable
term to describe Clang's AST. We will cover some specific cases later (TBD
link).}
with leaf nodes for different objects, such as function declarations and
loop bodies. Clang provides advanced tools for searching (matching) different
nodes, which are implemented in the form of a DSL (domain-specific language). It
is important to understand how it is implemented to be able to use it.

We will start with the basic data structures and class hierarchy that Clang uses
to create the AST. We will discuss the different methods used for AST traversal
and some helper classes for node matching during the traversal. 

\section{AST}
Typical tree structure implemented in C++ has all nodes derived from a base
class. Clang uses a different approach. It splits different C++ constructions
into separate groups with basic classes for each of them.
\begin{itemize}
  \item Statements. \mintinline{c++}{clang::Stmt} is the basic class for all
    statements. That includes 
    ordinary statements such as \mintinline{c++}{clang::IfStmt} as well as expressions
    and other C++ constructions
  \item Declarations. \mintinline{c++}{clang::Decl} is the basic class for
    declarations. This includes a variable, typedef, function, struct,
    etc. There is also a separate basic class for declarations with context
    i.e. declarations that might contain another declarations:
    \mintinline{c++}{clang::DeclContext}. Translation units
    (\mintinline{c++}{clang::TranslationUnitDecl} class) and namespaces
    (\mintinline{c++}{clang::NamespaceDecl} class) are typical examples for
    declarations with 
    context.  
  \item Types. \mintinline{c++}{clang::Type} is basic class for types representations.
\end{itemize}
Lets look at all the groups in details.

\subsection{Statements}
\mintinline{c++}{Stmt} is the basic class for all statements. The statements can
be combined into 2 sets (see \cref{fig:ast_stmt}). The first one contains
statements with values and an opposite group is for statements without values.
\begin{figure}[H]
  \resizebox{.9\linewidth}{!}{\
    \begin{forest}
      for tree={draw, rounded corners=5mm, font=\sffamily,
        minimum width=2cm, minimum height=1cm,
        forked edge, edge={blue,-latex},
        s sep=2cm, l sep=1cm, fork sep=5mm}
      [clang::Stmt
        [Statements without a value
          [clang::IfStmt]
          [clang::CompoundStmt]
          [Other statements]
        ]
        [Statements with a value
          [clang::ValueStmt]
        ]
      ] 
    \end{forest}
  }
  \caption{Clang AST: Statements}  
  \label{fig:ast_stmt}
\end{figure}

The group of statements without a value consist of different C++ constructions
such as if-statement (\mintinline{c++}{clang::IfStmt} class) or compound statement
(\mintinline{c++}{CompoundStmt} class). The majority of all statements falls
into the group.

The group of statement with a value consists of one base class
\mintinline{c++}{clang::ValueStmt} that has several children such as
\mintinline{c++}{clang::LabelStmt} (for labels representation) or
\mintinline{c++}{clang::ExprStmt} (for expressions representation), see
\cref{fig:ast_valuestmt}.

\begin{figure}
  \resizebox{.9\linewidth}{!}{\
    \begin{forest}
      for tree={draw, rounded corners=5mm, font=\sffamily,
        minimum width=2cm, minimum height=1cm,
        forked edge, edge={blue,-latex},
        s sep=2cm, l sep=1cm, fork sep=5mm}
      [clang::ValueStmt [
          clang::Expr [clang::BinaryOperator] [clang::CallExpr] [Other expressions]
        ]
        [clang::LabelStmt]
      ]
    \end{forest}
  }
  \caption{Clang AST: Statements with a value}  
  \label{fig:ast_valuestmt}
\end{figure}

\subsection{Declarations}
Declarations can also be combined into 2 primary groups: Declarations with
context and without. Declarations with context can be considered as placeholders
for other declarations. For example C++ namespace as well as translation unit or
function declaration might contain other declarations. Declaration of a friend
entity (\mintinline{c++}{clang::DeclFriend}) can 
be considered as an example of a declaration without context.

It has to be noted that classes that are inherited from
\mintinline{c++}{DeclContext} have also \mintinline{c++}{clang::Decl} as their
top parent.

Some declarations can be redeclarated as in the following example
\inputminted{c++}{./src/part1/ch2_arch/redeclaration.cpp}
Such declarations have an additional parent that is implemented via
\mintinline{c++}{clang::Redeclarable<...>} template.

\subsection{Types}
C++ is a statically typed language, which means that the types of
variables must be declared at compile-time. The types allow compiler to make a
reasonable conclusion about the program meaning i.e. types are important part of
semantic analysis. \mintinline{c++}{clang::Type} is the basic class for types in
Clang.

Types in C/C++ might have qualifiers that are called as CV-qualifiers at
standard \citep[basic.type.qualifier]{standard:cpp20}. CV here is for 2 keywords
\mintinline{c++}{const} and \mintinline{c++}{volatile} that can be used as the
qualifier for a type.
\footnote{
C99 standard has an additional type qualifier \mintinline{c++}{restrict} that is
also supported by clang \citep[6.7.3]{standard:c99} 
}
Clang has a special class to support a type with qualifier:
\mintinline{c++}{clang::QualType} that is a pair of a pointer to
\mintinline{c++}{clang::Type} and a bit mask with information about the type
qualifier. The class has a method to retrieve a pointer to the
\mintinline{c++}{clang::Type} and check different qualifiers. The code below
shows how we can check a type for a const qualifier
\footnote{The code is from LLVM release 16.0,
\texttt{clang/lib/AST/ExprConstant.cpp}, line 3855} 
% clang/lib/AST/ExprConstant.cpp:3855
\begin{minted}{c++}
  bool checkConst(QualType QT) {
    // Assigning to a const object has undefined behavior.
    if (QT.isConstQualified()) {
      Info.FFDiag(E, diag::note_constexpr_modify_const_type) << QT;
      return false;
    }
    return true;
  }
\end{minted}
It's worth mentioning that \mintinline{c++}{clang::QualType} has
\mintinline{c++}{operator->()} and \mintinline{c++}{operator*()} implemented,
i.e. it can be considered as a smart pointer for underlying
\mintinline{c++}{clang::Type} class.

In addition to the qualifiers type can have additional information that
represents different memory address models, for instance there can be a
pointer to an object or
reference. \mintinline{c++}{clang::Type}
has the following helper methods to check different address models:
\begin{itemize}
\item \mintinline{c++}{clang::Type::isPointerType()} for pointer type check
\item \mintinline{c++}{clang::Type::isReferenceType()} for reference type check
\end{itemize}

Types in C/C++ can also use aliases that are introduced by using
\mintinline{c++}{typedef} or \mintinline{c++}{using} keywords. The following
code defines \mintinline{c++}{foo} and \mintinline{c++}{bar} as aliases for
\mintinline{c++}{int} type
\begin{minted}{c++}
using foo = int;
typedef int bar;
\end{minted}
Original types, \mintinline{c++}{int} at our case, are called as canonical. You
can test if the type is canonical or not using
\mintinline{c++}{clang::QualType::isCanonical()}
method. \mintinline{c++}{clang::QualType} also provides a method to retrieve the
canonical type from an alias: \mintinline{c++}{clang::QualType::getCanonicalType()}.

\section{AST traversal. Test tool}
Compiler need to traverse AST as soon as its generated to produce IR code. Thus
having a suitable data structure for tree traversal is essential for AST design,
i.e. AST has to be designed to provide an easiest way for the tree
traversal. Typical way is to have common base class for AST nodes and the class
has to provide a method to retrieve the node's children. As we know Clang's AST
nodes do not have a common ancestor. How the tree traversal can be organized at
the situation?

To answer on the question we will create a simple clang tool that uses recursive
AST visitor for traversal. The tool will emit a message when it finds a
reference to a declaration (i.e. variable usage) in the analysed source code. We
will use the same CMake file as it was 
created for our first clang tool (see \cref{lis:cmake_sytax_check}), the only
one additional library is added for the new tool: \texttt{clangAST}. The result
\texttt{CMakeLists.txt} is the following:
\inputminted[highlightlines={19}, firstline=2,lastline=24]{cmake}{src/part1/ch3_ast/astdumper/CMakeLists.txt}

The \texttt{main} function for our tool looks like
\inputminted[highlightlines={5,23}]{c++}{src/part1/ch3_ast/astdumper/AstDumper.cpp}
As you may see (line 5,23), we use a custom frontend action for our project:
\texttt{clangbook::astdumper::FrontendAction}.

The code for the class is
\inputminted{c++}{src/part1/ch3_ast/astdumper/FrontendAction.hpp}
As you may see we redefined
\mintinline{c++}{clang::ASTFrontendAction::CreateASTConsumer} 
where we create our custom \mintinline{c++}{clangbook::astdumper::Consumer}(line 9).

The code for our custom AST consumer is the following:
\inputminted{c++}{src/part1/ch3_ast/astdumper/Consumer.hpp}

As you may see we create an example visitor and invoke it at
\mintinline{c++}{clang::ASTConsumer::HandleTranslationUnit}. The code for the
visior is the most interesting for us:
\inputminted{c++}{src/part1/ch3_ast/astdumper/Visitor.hpp}
The code will print a message with the found object name when it finds a
reference to a declaration, see lines 8-9. 

We can build our program to use the same sequence of commands as for our test
project, see \fullref{sec:setup_test_project}
\begin{adjustwidth}{0em}{0em}
\begin{verbatim}
export LLVM_HOME=<...>/llvm-project/install
mkdir build
cd build
cmake -G Ninja -DCMAKE_BUILD_TYPE=Debug ..
ninja
\end{verbatim}
\end{adjustwidth}
As you may notice we used \myshell{-DCMAKE\_BUILD\_TYPE=Debug} option
for CMake. That is because we want to investigate the result program under
debugger. 

The program we used for investigations previously (see \cref{fig:arch_max}) will
be used for AST traversal investigations as well:
\inputminted{c++}{./src/part1/ch2_arch/max.cpp}
The program has 4 references to declarations: variables 'a' and 'b' at line 2,
variable 'a' at line 3 and variable 'b' at line 4.

We can run our program as follows:
% ./build/astdumper ../../ch2_arch/max.cpp 
\begin{adjustwidth}{0em}{0em}
\begin{verbatim}
$ ./astdumper max.cpp 
...
Found reference to a declaration: 'a'
Found reference to a declaration: 'b'
Found reference to a declaration: 'a'
Found reference to a declaration: 'b'
\end{verbatim}
\end{adjustwidth}

Our test program uses \mintinline{c++}{clang::RecursiveASTVisitor} and our next
task will be to get a better understanding how it works. We will use \lldb
debugger as our primary tool for the investigations.

\section{Recursive AST Visitor}
As it was mentioned earlier we are going to use debugger for the recursive
visitor investigations. The debugger session can be run as follows
% lldb ./build/astdumper -- ../../ch2_arch/max.cpp
\begin{adjustwidth}{0em}{0em}
\begin{verbatim}
$ lldb ./astdumper -- max.cpp
\end{verbatim}
\end{adjustwidth}
We will look into the traversal that finds a function declaration. We can setup
breakpoint into
\mintinline{c++}{clangbook::astdumper::Visitor::VisitDeclRefExpr} and run the
debugging session:
\begin{adjustwidth}{0em}{0em}
\begin{verbatim}
(lldb) b clangbook::astdumper::Visitor::VisitDeclRefExpr
(lldb) r
...
* thread #1, name = 'astdumper', stop reason = breakpoint 1.1
    frame #0: ... ::Visitor::VisitDeclRefExpr(...) at Visitor.hpp:8:15
   5    class Visitor : public clang::RecursiveASTVisitor<Visitor> {
   6    public:
   7      bool VisitDeclRefExpr(const clang::DeclRefExpr *DRE) {
-> 8        llvm::errs() << "Found reference to a declaration: '"
   9                     << DRE->getFoundDecl()->getName() << "'\n";
   10       return true;
   11     }
(lldb)
\end{verbatim}
\end{adjustwidth}
We will start our investigation with 11th frame:
\begin{adjustwidth}{0em}{0em}
\begin{verbatim}
(lldb) f 11
frame #11: ... ::HandleTranslationUnit(...) at Consumer.hpp:11:20
   8      Consumer() : V(std::make_unique<Visitor>()) {}
   9    
   10     virtual void HandleTranslationUnit(clang::ASTContext &Context) override {
-> 11       V->TraverseDecl(Context.getTranslationUnitDecl());
   12     }
   13   
   14   private:
\end{verbatim}
\end{adjustwidth}
This is our code that starts the traversal procedure. We can use "up" and "down"
command to navigate the stack. The next frame will be frame \#10:
be
\begin{adjustwidth}{0em}{0em}
\begin{verbatim}
(lldb) down
frame #10: ...::TraverseDecl(...) at DeclNodes.inc:645:1
   642  #ifndef TRANSLATIONUNIT
   643  #  define TRANSLATIONUNIT(Type, Base) DECL(Type, Base)
   644  #endif
-> 645  TRANSLATIONUNIT(TranslationUnit, Decl)
   646  #undef TRANSLATIONUNIT
   647  
   648  LAST_DECL_RANGE(Decl, AccessSpec, TranslationUnit)
\end{verbatim}
\end{adjustwidth}
It reflects the fact that we start with translation unit traversal. Frame \#8
 gives us the code where we do the real traversal
\begin{adjustwidth}{0em}{0em}
\begin{verbatim}
(lldb) f 8
frame #8: ...::TraverseDeclContextHelper(...) at RecursiveASTVisitor.h:1489:7
   1486 
   1487   for (auto *Child : DC->decls()) {
   1488     if (!canIgnoreChildDeclWhileTraversingDeclContext(Child))
-> 1489       TRY_TO(TraverseDecl(Child));
   1490   }
   1491 
   1492   return true;
\end{verbatim}
\end{adjustwidth}
The \mintinline{c++}{clang::TranslationUnitDecl} is inherited from
\mintinline{c++}{clang::DeclContext} that is a holder for another
declarations. The class provides \mintinline{c++}{clang::DeclContext::decls}
method that can be used for a loop over all declarations stored at the specific
\mintinline{c++}{clang::DeclContext} instance. Our translation unit keeps
several declarations and one of them is our function. We can see it if we print
an AST node for the \mintinline{c++}{Child} variable that was selected
for execution at our breakpoint
\footnote{
It should be an AST node that corresponds to a declaration reference because we
stopped at the top frame on the method that processes function declarations
}
\begin{adjustwidth}{0em}{0em}
\begin{verbatim}
(lldb) p Child->dump()
FunctionDecl ... max.cpp:1:1, line:5:1> line:1:5 max 'int (int, int)'
|-ParmVarDecl ... used a 'int'
|-ParmVarDecl ... used b 'int'
`-CompoundStmt ... <col:23, line:5:1>
...
\end{verbatim}
\end{adjustwidth}

It's worth mentioning that there is a common pattern for AST recursive
visitor. It uses ad-hoc methods of AST nodes for the tree traversal. For
instance it uses loop over declarations stored at translation unit for finding a
declaration reference.

%% First of all, each AST node has ad-hoc methods that allow to select different
%% children. For example \mintinline{c++}{clang::IfStmt} has a method to select
%% conditional expression: \mintinline{c++}{clang::IfStmt::getCond()}, method to
%% select \texttt{then} statement: \mintinline{c++}{clang::IfStmt::getThen()} and a
%% method to select \texttt{else} statement
%% \mintinline{c++}{clang::IfStmt::getElse()}.

%% Having different methods to retrieve children does not make our life easiest. 
%% TBD

\section{AST matchers}
TBD

\section{Explore clang AST with clang-query}

TBD

\section{Errors processing}
\clang tries to collect as many errors as possible while compiling a program.
TBD

\section{Summary}

TBD


\section{Further reading}
\begin{itemize}
  \item How to write RecursiveASTVisitor: https://clang.llvm.org/docs/RAVFrontendAction.html
\end{itemize}


