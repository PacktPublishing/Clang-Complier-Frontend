%% -*- coding:utf-8 -*-
\chapter{Clang AST}
The Abstract Syntax Tree (AST) is a fundamental algorithmic structure used to
represent the results of parsing. The AST serves as the framework for the Clang frontend and is the primary tool
for various Clang utilities, including linters. Clang offers sophisticated tools
for searching (or matching) various AST nodes. These tools are implemented using
a DSL (domain-specific language). It's crucial to understand its implementation
to use it effectively. 

We will start with the basic data structures and the class hierarchy that Clang
uses to construct the AST. Additionally, we will explore the methods used for
AST traversal and highlight some helper classes that facilitate node matching
during this traversal. 

\section{AST}
The AST is usually
depicted as a tree, with its leaf nodes corresponding to various objects, such
as function declarations and loop bodies. 
Typically, the AST represents the result of syntax analysis, i.e.,
parsing. Clang's AST nodes were designed to be immutable. This design requires
that the Clang AST store results from semantic analysis, meaning the Clang AST
represents the outcomes of both syntax and semantic analyses. 

\begin{quote}
Although Clang also
employs an AST, it's worth noting that the Clang AST is not a true tree. The
presence of backward edges makes "graph" a more appropriate term for describing
Clang's AST.
\end{quote}

Typical tree structure implemented in C++ has all nodes derived from a base
class. Clang uses a different approach. It splits different C++ constructions
into separate groups with basic classes for each of them.
\begin{itemize}
  \item Statements. \mintinline{c++}{clang::Stmt} is the basic class for all
    statements. That includes 
    ordinary statements such as if-statement (\mintinline{c++}{clang::IfStmt}
    class) as well as expressions and other C++ constructions
  \item Declarations. \mintinline{c++}{clang::Decl} is the basic class for
    declarations. This includes a variable, typedef, function, struct,
    etc. There is also a separate basic class for declarations with context
    i.e. declarations that might contain another declarations:
    \mintinline{c++}{clang::DeclContext}. Translation units
    (\mintinline{c++}{clang::TranslationUnitDecl} class) and namespaces
    (\mintinline{c++}{clang::NamespaceDecl} class) are typical examples for
    declarations with 
    context.  
  \item Types. C++ has rich type system. It includes basic types such as
    \mintinline{c++}{int} for integers as well as custom defined types and types
    redefinition via \mintinline{c++}{typedef} or \mintinline{c++}{using}. Types
    in C++ can have qualifiers such as \mintinline{c++}{const} and can represent
    different memory addressing modes aka pointers, references etc. Clang uses
    \mintinline{c++}{clang::Type} as the basic class for types representations
    in AST. 
\end{itemize}
It's worth noting that there are additional relations between the groups. For
example, the \mintinline{c++}{clang::DeclStmt} class, that inherits from
\mintinline{c++}{clang::Stmt}, has methods to retrieve corresponding
declarations. Additionally, expressions (represented by the
\mintinline{c++}{clang::clang::Expr} class) which inherit from
\mintinline{c++}{clang::Stmt} have methods to work with types. Let's look at all
the groups in detail. 

\subsection{Statements}
\mintinline{c++}{Stmt} is the basic class for all statements. The statements can
be combined into two sets (see \cref{fig:ast_stmt}). The first one contains
statements with values and an opposite group is for statements without values.
\begin{figure}[H]
  \resizebox{.9\linewidth}{!}{\
    \begin{forest}
      for tree={draw, rounded corners=5mm, font=\sffamily,
        minimum width=2cm, minimum height=1cm,
        forked edge, edge={blue,-latex},
        s sep=2cm, l sep=1cm, fork sep=5mm}
      [clang::Stmt
        [Statements without a value
          [clang::IfStmt]
          [clang::CompoundStmt]
          [Other statements]
        ]
        [Statements with a value
          [clang::ValueStmt]
        ]
      ] 
    \end{forest}
  }
  \caption{Clang AST: Statements}  
  \label{fig:ast_stmt}
\end{figure}

The group of statements without a value consist of different C++ constructions
such as if-statement (\mintinline{c++}{clang::IfStmt} class) or compound statement
(\mintinline{c++}{CompoundStmt} class). The majority of all statements falls
into the group.

The group of statement with a value consists of one base class
\mintinline{c++}{clang::ValueStmt} that has several children such as
\mintinline{c++}{clang::LabelStmt} (for labels representation) or
\mintinline{c++}{clang::ExprStmt} (for expressions representation), see
\cref{fig:ast_valuestmt}.

\begin{figure}
  \resizebox{.9\linewidth}{!}{\
    \begin{forest}
      for tree={draw, rounded corners=5mm, font=\sffamily,
        minimum width=2cm, minimum height=1cm,
        forked edge, edge={blue,-latex},
        s sep=2cm, l sep=1cm, fork sep=5mm}
      [clang::ValueStmt [
          clang::Expr [clang::BinaryOperator] [clang::CallExpr] [Other expressions]
        ]
        [clang::LabelStmt]
      ]
    \end{forest}
  }
  \caption{Clang AST: Statements with a value}  
  \label{fig:ast_valuestmt}
\end{figure}

\subsection{Declarations}
Declarations can also be combined into 2 primary groups: Declarations with
context and without. Declarations with context can be considered as placeholders
for other declarations. For example C++ namespace as well as translation unit or
function declaration might contain other declarations. Declaration of a friend
entity (\mintinline{c++}{clang::DeclFriend}) can 
be considered as an example of a declaration without context.

It has to be noted that classes that are inherited from
\mintinline{c++}{DeclContext} have also \mintinline{c++}{clang::Decl} as their
top parent.

Some declarations can be redeclarated as in the following example
\inputminted{c++}{./src/part1/ch2_arch/redeclaration.cpp}
Such declarations have an additional parent that is implemented via
\mintinline{c++}{clang::Redeclarable<...>} template.

\subsection{Types}
C++ is a statically typed language, which means that the types of
variables must be declared at compile-time. The types allow compiler to make a
reasonable conclusion about the program meaning i.e. types are important part of
semantic analysis. \mintinline{c++}{clang::Type} is the basic class for types in
Clang.

Types in C/C++ might have qualifiers that are called as CV-qualifiers at
standard \citep[basic.type.qualifier]{standard:cpp20}. CV here is for 2 keywords
\mintinline{c++}{const} and \mintinline{c++}{volatile} that can be used as the
qualifier for a type.
\begin{quote}
C99 standard has an additional type qualifier \mintinline{c++}{restrict} that is
also supported by clang \citep[6.7.3]{standard:c99} 
\end{quote}
Clang has a special class to support a type with qualifier:
\mintinline{c++}{clang::QualType} that is a pair of a pointer to
\mintinline{c++}{clang::Type} and a bit mask with information about the type
qualifier. The class has a method to retrieve a pointer to the
\mintinline{c++}{clang::Type} and check different qualifiers. The code below
shows how we can check a type for a const qualifier (the code is from LLVM
release 16.0, \texttt{clang/lib/AST/ExprConstant.cpp}, line 3855)
% clang/lib/AST/ExprConstant.cpp:3855
\begin{minted}{c++}
  bool checkConst(QualType QT) {
    // Assigning to a const object has undefined behavior.
    if (QT.isConstQualified()) {
      Info.FFDiag(E, diag::note_constexpr_modify_const_type) << QT;
      return false;
    }
    return true;
  }
\end{minted}
It's worth mentioning that \mintinline{c++}{clang::QualType} has
\mintinline{c++}{operator->()} and \mintinline{c++}{operator*()} implemented,
i.e. it can be considered as a smart pointer for underlying
\mintinline{c++}{clang::Type} class.

In addition to the qualifiers type can have additional information that
represents different memory address models, for instance there can be a
pointer to an object or
reference. \mintinline{c++}{clang::Type}
has the following helper methods to check different address models:
\begin{itemize}
\item \mintinline{c++}{clang::Type::isPointerType()} for pointer type check
\item \mintinline{c++}{clang::Type::isReferenceType()} for reference type check
\end{itemize}

Types in C/C++ can also use aliases that are introduced by using
\mintinline{c++}{typedef} or \mintinline{c++}{using} keywords. The following
code defines \mintinline{c++}{foo} and \mintinline{c++}{bar} as aliases for
\mintinline{c++}{int} type
\begin{minted}{c++}
using foo = int;
typedef int bar;
\end{minted}
Original types, \mintinline{c++}{int} at our case, are called as canonical. You
can test if the type is canonical or not using
\mintinline{c++}{clang::QualType::isCanonical()}
method. \mintinline{c++}{clang::QualType} also provides a method to retrieve the
canonical type from an alias: \mintinline{c++}{clang::QualType::getCanonicalType()}.

\section{AST traversal}
\label{sec:ch3:ast_traversal}
The compiler requires traversal of the AST to generate IR code. Thus, having a
well-structured data structure for tree traversal is paramount for AST
design. To put it another way, the design of the AST should prioritize
facilitating easy tree traversal. A standard approach in many systems is to have
a common base class for all AST nodes. This class typically provides a method to
retrieve the node's children, allowing for tree traversal using popular
algorithms like Breadth-First Search (BFS)
\cite{book:cormen2009introduction}. Clang, however, takes a different approach:
its AST nodes don't share a common ancestor. This poses the question: how is
tree traversal organized in Clang? 

Clang employs three unique techniques:
\begin{itemize}
\item The Curiously Recurring Template Pattern (CRTP) for visitor class definition,
\item Ad-hoc methods tailored specifically for different nodes,
\item Macros, which can be perceived as the connecting layer between the ad-hoc methods and CRTP.
\end{itemize}

We will explore these techniques through a simple program  designed to identify
function definitions and display the function names together with their
parameters.  

\subsection{DeclVisitor test tool}
Our test tool will build upon the \mintinline{c++}{clang::DeclVisitor} class,
which is defined as a straightforward visitor class aiding in the creation of
visitors for C/C++ declarations. 

We will use the same CMake file as it was 
created for our first clang tool (see \cref{lis:cmake_sytax_check}). The sole
addition for the new tool is the \texttt{clangAST} library. The resultant
\texttt{CMakeLists.txt} is shown in \cref{lis:cmake:declvisitor}:
\begin{figure}[H]
\inputminted[highlightlines={19},
  firstline=2,lastline=24]{cmake}{src/part1/ch3_ast/declvisitor/CMakeLists.txt}
\caption{CMakeLists.txt file for DeclVisitor test tool}
\label{lis:cmake:declvisitor}
\end{figure}

The \mintinline{c++}{main} function of our tool is presented below:
\begin{figure}[H]
  \inputminted[highlightlines={5,23}]{c++}{src/part1/ch3_ast/declvisitor/DeclVisitor.cpp}
  \caption{The main function for DeclVisitor test tool}
  \label{lis:main:declvisitor}
\end{figure}

From lines 5 and 23, it's evident that we employ a custom frontend action
specific to our project:
\mintinline{c++}{clangbook::declvisitor::FrontendAction}. The code for this
class is provided below
\begin{figure}[H]
\inputminted[highlightlines={9-12}]{c++}{src/part1/ch3_ast/declvisitor/FrontendAction.hpp}
  \caption{Custom FrontendAction class for DeclVisitor test tool}
  \label{lis:frontendaction:declvisitor}
\end{figure}

You'll notice that we have overridden the
\mintinline{c++}{clang::ASTFrontendAction::CreateASTConsumer} function to
instantiate our custom \mintinline{c++}{clangbook::declvisitor::Consumer} (as
highlighted in lines 9-12). The implementation for our tailored AST consumer is as
follows:
\begin{figure}[H]
\inputminted{c++}{src/part1/ch3_ast/declvisitor/Consumer.hpp}
\caption{Consumer class for DeclVisitor test tool}
\label{lis:consumer:declvisitor}
\end{figure}

Here, we can see that we've created a sample visitor and invoked it using the \\
\mintinline{c++}{clang::ASTConsumer::HandleTranslationUnit} (see
\cref{lis:consumer:declvisitor}, line 11). However, the most
intriguing portion is the code for the visitor: 
\begin{figure}[H]
  \inputminted[highlightlines={8,14}]{c++}{src/part1/ch3_ast/declvisitor/Visitor.hpp}
  \caption{Visitor class implementation}
  \label{lis:declvisitor:visitor}
\end{figure}
We will explore the code in more depth later. For now, we observe that it prints
the function name at line 8 and the parameter name at line 14. 

We can compile our program using the same sequence of commands as we did for our
test project, as detailed in \fullref{sec:setup_test_project}. 
% cd src/part1/ch3_ast/declvisitor
% ./build.sh
\begin{figure}[H]
\begin{adjustwidth}{0em}{0em}
\begin{verbatim}
export LLVM_HOME=<...>/llvm-project/install
mkdir build
cd build
cmake -G Ninja -DCMAKE_BUILD_TYPE=Debug ..
ninja
\end{verbatim}
\end{adjustwidth}
\caption{Configure and build commands for DeclVisitor test tool}
\label{lis:declvisitor:cmakeconfig}
\end{figure}
As you may notice we used \myshell{-DCMAKE\_BUILD\_TYPE=Debug} option
for CMake. That is because we might want to investigate the result program under
debugger.

We will use the same program we referenced in our previous investigations (see \cref{fig:arch_max}) to also study AST traversal:
\begin{figure}[H]
\inputminted{c++}{./src/part1/ch2_arch/max.cpp}
\caption{Test program max.cpp}
\label{lis:declvisitor:max}
\end{figure}
This program consists of a single function, \mintinline{c++}{max}, which takes
two parameters: \mintinline{c++}{a} and \mintinline{c++}{b}, and returns the
maximum of the two. 

We can run our program as follows:
% ./build/declvisitor ../../ch2_arch/max.cpp 
\begin{adjustwidth}{0em}{0em}
\begin{verbatim}
$ ./declvisitor max.cpp 
...
Function: 'max'
        Parameter: 'a'
        Parameter: 'b'
\end{verbatim}
\end{adjustwidth}
Let's investigate the Visitor class implementation in detail.

\subsection{Visitor implementation}
\label{sec:part1:ch3:visitor_implementation}
Let's delve into the Visitor code (see \cref{lis:declvisitor:visitor}).
Firstly, you'll notice an unusual construct where our class is derived from a
base class parameterized by our own class: 
\inputminted[firstline=5,lastline=5]{c++}{src/part1/ch3_ast/declvisitor/Visitor.hpp}
This construct is known as the Curiously Recurring Template Pattern, or CRTP.

The Visitor class has several callbacks that are triggered when a corresponding
AST node is encountered. The first callback targets the AST node representing a
function declaration
\begin{figure}[H]
  \inputminted[firstline=7,lastline=12]{c++}{src/part1/ch3_ast/declvisitor/Visitor.hpp}
  \caption{FunctionDecl callback}
  \label{lis:declvisitor:visitor:visitfunctiondecl}
\end{figure}
As shown in \cref{lis:declvisitor:visitor:visitfunctiondecl}, the function name
is printed on line 8. Our subsequent step involves printing the names of the
parameters. To retrieve the function parameters, we can utilize the
\mintinline{c++}{parameters()} method from the
\mintinline{c++}{clang::FunctionDecl} class. This method was previously
mentioned as an ad-hoc approach for AST traversal. Each AST node provides its
own methods to access child nodes. Since we have an AST node of a specific type
(i.e., \mintinline{c++}{clang::FunctionDecl*}) as an argument, we can employ
these methods. 

The function parameter is passed to the \mintinline{c++}{Visit(...)} method of
the base class \mintinline{c++}{clang::DeclVisitor<>}. This call is subsequently
transformed into another callback, specifically for the
\mintinline{c++}{clang::ParmVarDecl} AST node. 
\begin{figure}[H]
  \inputminted[firstline=13,lastline=15]{c++}{src/part1/ch3_ast/declvisitor/Visitor.hpp}
  \caption{ParmVarDecl callback}
  \label{lis:declvisitor:visitor:visitparmvardecl}
\end{figure}
You might be wondering how this conversion is achieved. The answer lies in a
combination of the CRTP and C/C++ macros. To understand this, we need to dive
into the \mintinline{c++}{Visit()} method implementation of the
\mintinline{c++}{clang::DeclVisitor<>} class. This implementation heavily relies
on C/C++ macros, so to get a glimpse of the actual code, we must expand these
macros. This can be done using the \mintinline{shell}{-E} compiler option. Let's
make some modifications to our \myshell{CMakeLists.txt} and introduce a new
custom target.  
\begin{figure}[H]
\inputminted[highlightlines={19},
  firstline=25,lastline=31]{cmake}{src/part1/ch3_ast/declvisitor/CMakeLists.txt}
\caption{Custom target to expand macros}
\label{lis:declvisitor:cmakelists:customtarget}
\end{figure}
We can run the target as follows
\begin{adjustwidth}{0em}{0em}
\begin{verbatim}
ninja preprocess
\end{verbatim}
\end{adjustwidth}
The resulting file can be located in the build folder specified earlier, named
\\ \myshell{DeclVisitor.cpp.preprocessed}.
This is the folder we pointed to when
executing the cmake command (see \cref{lis:declvisitor:cmakeconfig}). Within
this file, the generated code for the \mintinline{c++}{Visit()} method appears
as follows:  
\begin{minted}{cpp}
RetTy Visit(typename Ptr<Decl>::type D) {
  switch (D->getKind()) {
    ...
    case Decl::ParmVar: return static_cast<ImplClass*>(this)->VisitParmVarDecl(static_cast<typename Ptr<ParmVarDecl>::type>(D)); 
    ...
  }
}
\end{minted}
This code showcases the use of the CRTP
in Clang. In this context, CRTP is employed to redirect back to our
\mintinline{c++}{Visitor} class, which is referenced as
\mintinline{c++}{ImplClass}. CRTP allows the base class to call a method from an
inherited class. This pattern can serve as an alternative to virtual functions
and offers several advantages, the most notable being
performance-related. Specifically, the method call is resolved at compile-time,
eliminating the need for a vtable lookup associated with virtual method calls. 

The code was generated using C/C++ macros, as demonstrated below. This
particular code was sourced from the \myshell{clang/include/clang/AST/DeclVisitor.h} header.
\begin{minted}{c++}
  #define DISPATCH(NAME, CLASS) \
  return static_cast<ImplClass*>(this)->Visit##NAME(static_cast<PTR(CLASS)>(D))
\end{minted}
The \mintinline{c++}{NAME} is replaced with the node name; in our case, it's
\mintinline{c++}{ParmVarDecl}. 

The \mintinline{c++}{DeclVisitor} is used to traverse C++ declarations. Clang
also has \mintinline{c++}{StmtVisitor} and \mintinline{c++}{TypeVisitor} to
traverse statements and types, respectively. These are built on the same
principles as we  considered in our example with the declaration
visitor. However, these visitors come with several 
issues. They can only be used with specific groups of AST nodes. For instance,
the \mintinline{c++}{DeclVisitor} can only be used with descendants of the
\mintinline{c++}{Decl} class. 
Another limitation is that we are required to implement recursion. For example,
we set up recursion to traverse the function declaration in lines 
9-11 (\cref{lis:declvisitor:visitor}). The same recursion was employed to
traverse declarations within the translation unit (see
\cref{lis:declvisitor:visitor}, lines 17-19). This brings up another concern:
it's possible to miss some parts of the recursion. For instance, our code will
not function correctly if the \mintinline{c++}{max} function declaration is
specified inside a namespace. 
%./build/declvisitor  ../../ch2_arch/maxns.cpp
To address such scenarios, we would need to implement an additional visit method
specifically for namespace declarations.

These challenges are addressed by the recursive visitor, which we will discuss
shortly. 

\section{Recursive AST Visitor}
\label{sec:ch3:recursive_ast_visitor}
Recursive AST visitors address the limitations observed with specialized
visitors. We will create the same program, which searches for and prints
function declarations along with their parameters, but we'll use a recursive
visitor this time. 

The \myshell{CMakeLists.txt} for recursive visitor test tool will be similar to
used previously. The only project name and source file name was changed
\begin{figure}[H]
\inputminted[highlightlines={2,14-17}]{cmake}{src/part1/ch3_ast/recursivevisitor/CMakeLists.txt}
\caption{CMakeLists.txt file for RecursiveVisitor test tool}
\label{lis:cmake:recursivevisitor}
\end{figure}

The \mintinline{c++}{main} function for our tool is similar to the `DeclVisitor`
one defined in \cref{lis:main:declvisitor}. 
\begin{figure}[H]
\inputminted[highlightlines={23}]{c++}{src/part1/ch3_ast/recursivevisitor/RecursiveVisitor.cpp}
\caption{The main function for the RecursiveVisitor test tool.}
\label{lis:main:recursivevisitor}
\end{figure}
As you can see, we changed only the namespace name for our custom frontend
action at line 23. 

The code for the frontend action and consumer is the same as in
\cref{lis:frontendaction:declvisitor} and \cref{lis:consumer:declvisitor}, with
the only difference being the namespace change from
\mintinline{c++}{declvisitor} to \mintinline{c++}{recursivevisitor}. The most
interesting part of the program is the Visitor class implementation. 
\begin{figure}[H]
\inputminted{c++}{src/part1/ch3_ast/recursivevisitor/Visitor.hpp}
\caption{Visitor class implementation}
  \label{lis:recursivevisitor:visitor}
\end{figure}
There are several changes compared to the code for `DeclVisitor` (see
\cref{lis:declvisitor:visitor}). The first is that recursion isn't
implemented. We've only implemented the callbacks for nodes of interest to us. A
reasonable question arises: how is the recursion controlled? The answer lies in
another change: our callbacks now return a boolean result. The value
\mintinline{c++}{false} indicates that the recursion should stop, while
\mintinline{c++}{true} signals the visitor to continue the traversal. 

The program can be compiled using the same sequence of commands as we used
previously, see \cref{lis:declvisitor:cmakeconfig}.

We can run our program as follows:
% ./build/recursivevisitor ../../ch2_arch/max.cpp 
\begin{adjustwidth}{0em}{0em}
\begin{verbatim}
$ ./recursivevisitor max.cpp 
...
Function: 'max'
        Parameter: 'a'
        Parameter: 'b'
\end{verbatim}
\end{adjustwidth}
As we can see, it produces the same result as we obtained with the DeclVisitor
implementation. The AST traversal techniques considered so far are not the only
ways for AST traversal. Most of the tools that we will consider later will use a
different approach based on AST matchers. 

%% Our test program uses \mintinline{c++}{clang::RecursiveASTVisitor} and our next
%% task will be to get a better understanding how it works. We will use \lldb
%% debugger as our primary tool for the investigations.

%% As it was mentioned earlier we are going to use debugger for the recursive
%% visitor investigations. The debugger session can be run as follows
%% % lldb ./build/astdumper -- ../../ch2_arch/max.cpp
%% \begin{adjustwidth}{0em}{0em}
%% \begin{verbatim}
%% $ lldb ./astdumper -- max.cpp
%% \end{verbatim}
%% \end{adjustwidth}
%% We will look into the traversal that finds a function declaration. We can setup
%% breakpoint into
%% \mintinline{c++}{clangbook::astdumper::Visitor::VisitDeclRefExpr} and run the
%% debugging session:
%% \begin{adjustwidth}{0em}{0em}
%% \begin{verbatim}
%% (lldb) b clangbook::astdumper::Visitor::VisitDeclRefExpr
%% (lldb) r
%% ...
%% * thread #1, name = 'astdumper', stop reason = breakpoint 1.1
%%     frame #0: ... ::Visitor::VisitDeclRefExpr(...) at Visitor.hpp:8:15
%%    5    class Visitor : public clang::RecursiveASTVisitor<Visitor> {
%%    6    public:
%%    7      bool VisitDeclRefExpr(const clang::DeclRefExpr *DRE) {
%% -> 8        llvm::errs() << "Found reference to a declaration: '"
%%    9                     << DRE->getFoundDecl()->getName() << "'\n";
%%    10       return true;
%%    11     }
%% (lldb)
%% \end{verbatim}
%% \end{adjustwidth}
%% We will start our investigation with 11th frame:
%% \begin{adjustwidth}{0em}{0em}
%% \begin{verbatim}
%% (lldb) f 11
%% frame #11: ... ::HandleTranslationUnit(...) at Consumer.hpp:11:20
%%    8      Consumer() : V(std::make_unique<Visitor>()) {}
%%    9    
%%    10     virtual void HandleTranslationUnit(clang::ASTContext &Context) override {
%% -> 11       V->TraverseDecl(Context.getTranslationUnitDecl());
%%    12     }
%%    13   
%%    14   private:
%% \end{verbatim}
%% \end{adjustwidth}
%% This is our code that starts the traversal procedure. We can use "up" and "down"
%% command to navigate the stack. The next frame will be frame \#10:
%% be
%% \begin{adjustwidth}{0em}{0em}
%% \begin{verbatim}
%% (lldb) down
%% frame #10: ...::TraverseDecl(...) at DeclNodes.inc:645:1
%%    642  #ifndef TRANSLATIONUNIT
%%    643  #  define TRANSLATIONUNIT(Type, Base) DECL(Type, Base)
%%    644  #endif
%% -> 645  TRANSLATIONUNIT(TranslationUnit, Decl)
%%    646  #undef TRANSLATIONUNIT
%%    647  
%%    648  LAST_DECL_RANGE(Decl, AccessSpec, TranslationUnit)
%% \end{verbatim}
%% \end{adjustwidth}
%% It reflects the fact that we start with translation unit traversal. Frame \#8
%%  gives us the code where we do the real traversal
%% \begin{adjustwidth}{0em}{0em}
%% \begin{verbatim}
%% (lldb) f 8
%% frame #8: ...::TraverseDeclContextHelper(...) at RecursiveASTVisitor.h:1489:7
%%    1486 
%%    1487   for (auto *Child : DC->decls()) {
%%    1488     if (!canIgnoreChildDeclWhileTraversingDeclContext(Child))
%% -> 1489       TRY_TO(TraverseDecl(Child));
%%    1490   }
%%    1491 
%%    1492   return true;
%% \end{verbatim}
%% \end{adjustwidth}
%% The \mintinline{c++}{clang::TranslationUnitDecl} is inherited from
%% \mintinline{c++}{clang::DeclContext} that is a holder for another
%% declarations. The class provides \mintinline{c++}{clang::DeclContext::decls}
%% method that can be used for a loop over all declarations stored at the specific
%% \mintinline{c++}{clang::DeclContext} instance. Our translation unit keeps
%% several declarations and one of them is our function. We can see it if we print
%% an AST node for the \mintinline{c++}{Child} variable that was selected
%% for execution at our breakpoint
%% \footnote{
%% It should be an AST node that corresponds to a declaration reference because we
%% stopped at the top frame on the method that processes function declarations
%% }
%% \begin{adjustwidth}{0em}{0em}
%% \begin{verbatim}
%% (lldb) p Child->dump()
%% FunctionDecl ... max.cpp:1:1, line:5:1> line:1:5 max 'int (int, int)'
%% |-ParmVarDecl ... used a 'int'
%% |-ParmVarDecl ... used b 'int'
%% `-CompoundStmt ... <col:23, line:5:1>
%% ...
%% \end{verbatim}
%% \end{adjustwidth}

%% It's worth mentioning that there is a common pattern for AST recursive
%% visitor. It uses ad-hoc methods of AST nodes for the tree traversal. For
%% instance it uses loop over declarations stored at translation unit for finding a
%% declaration reference.

%% First of all, each AST node has ad-hoc methods that allow to select different
%% children. For example \mintinline{c++}{clang::IfStmt} has a method to select
%% conditional expression: \mintinline{c++}{clang::IfStmt::getCond()}, method to
%% select \texttt{then} statement: \mintinline{c++}{clang::IfStmt::getThen()} and a
%% method to select \texttt{else} statement
%% \mintinline{c++}{clang::IfStmt::getElse()}.

%% Having different methods to retrieve children does not make our life easiest. 
%% TBD

\section{AST matchers}
AST matchers provide another approach for locating specific AST nodes. They can
be particularly useful in linters when searching for improper pattern usage or
in refactoring tools when identifying AST nodes for modification.

We will create a simple program to test AST matchers. The program will identify
a function definition with the name \mintinline{c++}{max}. 

We will use a slightly modified \myshell{CMakeLists.txt} file from the previous examples.
\begin{figure}[H]
  \inputminted[highlightlines={18,21}]{cmake}{src/part1/ch3_ast/matchvisitor/CMakeLists.txt}
  \caption{CMakeLists.txt for AST matchers test tool}
  \label{lis:ch3:matchervisitor:cmake}
\end{figure}
There are two additional libraries were added: \myshell{LLVMFrontendOpenMP} and
\myshell{clangASTMatchers} (see lines 18 and 21 in
\cref{lis:ch3:matchervisitor:cmake}).

The \mintinline{c++}{main} function for our tool looks like
\begin{figure}[H]
  \inputminted[highlightlines={5,22-24}]{c++}{src/part1/ch3_ast/matchvisitor/MatchVisitor.cpp}
  \caption{The main function for AST matchers test tool}
  \label{lis:ch3:matchervisitor:main}
\end{figure}
As you can observe (lines 22-24), we employ the \mintinline{c++}{MatchFinder}
class and define a custom callback (included via the header in line 5) that
outlines the specific AST node we intend to match. 

The callback is implemented as follows:
\begin{figure}[H]
\inputminted[highlightlines={7-9}]{c++}{src/part1/ch3_ast/matchvisitor/MatchCallback.hpp}
  \caption{The match callback for AST matchers test tool}
  \label{lis:ch3:matchervisitor:callback}
\end{figure}
The most crucial section of the code is located at lines 7-9. Each matcher is
identified by an ID, which, in our case, is 'match-id'. The matcher itself is
defined in lines 8-9: 
\inputminted[firstline=8,lastline=9]{c++}{src/part1/ch3_ast/matchvisitor/MatchCallback.hpp}
This matcher seeks a function declaration using \mintinline{c++}{functionDecl()} that has a specific name, as seen in \mintinline{c++}{matchesName()}.
We utilized a specialized Domain-Specific Language (DSL) to specify the
matcher. The DSL is implemented using C++ macros. It's worth noting that the
recursive AST visitor serves as the backbone for AST traversal inside the
matchers implementation.

The program can be compiled using the same sequence of commands as we used
previously, see \cref{lis:declvisitor:cmakeconfig}.

We will utilize a slightly modified version of the example shown in
\cref{fig:arch_max}, with an additional function added.
\begin{figure}[H]
  \inputminted{c++}{src/part1/ch3_ast/minmax.cpp}
  \caption{Test program minmax.cpp for AST matchers}
  \label{lis:min:max}
\end{figure}

When we run our test tool on the example, we will obtain the following output:
%./build/matchvisitor ../minmax.cpp
\begin{adjustwidth}{0em}{0em}
\begin{verbatim}
./matchvisitor minmax.cpp
...
Found 'max' function at minmax.cpp:1:5
\end{verbatim}
\end{adjustwidth}
As we can see, it has located only one function declaration with the name
specified for the matcher. 

The AST matchers reference page
(https://clang.llvm.org/docs/LibASTMatchersReference.html) provides extensive
information about various matchers and their usage. 

The DSL for matchers is typically employed in custom Clang tools, such as
clang-tidy (as discussed in \fullref{ch:clangtidy}), but it can also be used as
a standalone tool. A specialized program called \myshell{clang-query} enables
the execution of different match queries. 

\section{Explore clang AST with clang-query}
AST matchers are incredibly useful, and there's a utility that facilitates
checking various matchers and analyzing the AST of your source code. This
utility is known as \myshell{clang-query}. You can build and install this
utility using the following command. 
\begin{adjustwidth}{0em}{0em}
\begin{verbatim}
ninja install-clang-query
\end{verbatim}
\end{adjustwidth}
You can run the tool as follows
%./llvm-project-16.x/install/bin/clang-query ./src/part1/ch3_ast/minmax.cpp
\begin{adjustwidth}{0em}{0em}
\begin{verbatim}
llvm-project/install/bin/clang-query minmax.cpp
\end{verbatim}
\end{adjustwidth}
We can use \myshell{match} command as follows

\begin{adjustwidth}{0em}{0em}
\begin{verbatim}
clang-query> match functionDecl(decl().bind("match-id"), matchesName("max"))

Match #1:

minmax.cpp:1:1: note: "match-id" binds here
int max(int a, int b) {
^~~~~~~~~~~~~~~~~~~~~~~
minmax.cpp:1:1: note: "root" binds here
int max(int a, int b) {
^~~~~~~~~~~~~~~~~~~~~~~
1 match.
clang-query> 
\end{verbatim}
\end{adjustwidth}

The default output, referred to as \myshell{diag}, is available. Among several
potential outputs, the most relevant one for us is \myshell{dump}. When the
output is set to \myshell{dump}, clang-query will display the located AST
node. For example, the following demonstrates how to match a function parameter
named \myshell{a}: 

\begin{adjustwidth}{0em}{0em}
\begin{verbatim}
clang-query> set output dump
clang-query> match parmVarDecl(hasName("a"))

Match #1:

Binding for "root":
ParmVarDecl 0x6775e48 <minmax.cpp:1:9, col:13> col:13 used a 'int'


Match #2:

Binding for "root":
ParmVarDecl 0x6776218 <minmax.cpp:6:9, col:13> col:13 used a 'int'

2 matches.
clang-query>
\end{verbatim}
\end{adjustwidth}

This tool proves useful when you wish to test a particular matcher or
investigate a portion of the AST tree. We will utilize this tool to explore
how Clang handles compilation errors. 

\section{Processing AST in the case of errors}
One of the most interesting aspects of Clang pertains to error processing. Error
processing encompasses error detection, the display of corresponding error
messages, and potential error recovery. The latter is particularly intriguing in
terms of the Clang AST. Error recovery occurs when Clang doesn't halt upon
encountering a compilation error but continues to compile in order to detect
additional issues. 

Such behavior is beneficial for various reasons. The most evident one is user
convenience. When programmers compile a program, they typically prefer to be
informed about as many errors as possible in a single compilation run. If the
compiler were to stop at the first error, the programmer would have to correct
that error, recompile, then address the subsequent error, and recompile again,
and so forth. This iterative process can be tedious and frustrating, especially
with larger codebases or intricate errors. 

Another compelling reason centers on IDE integration, which will be discussed in
more detail in \fullref{ch:clangd}. IDEs offer navigation support coupled with
an integrated compiler. We will explore \myshell{clangd} as one of such
tools. Editing code in IDEs often leads to compilation errors. Most errors are
confined to specific sections of the code, and it might be suboptimal to cease
navigation in such cases. 

Clang employs various techniques for error recovery. For the syntax stage of
parsing, it utilizes heuristics; for instance, if a user forgets to insert a
semicolon, Clang may attempt to add it as part of the recovery process. The
semantic stage of parsing is more intricate, and Clang implements unique
techniques to manage recovery during this phase. 

Consider a program (maxerr.cpp) that is syntactically correct but has a semantic
error. For example, it might use an undeclared variable. In this program, refer
to line 3 where the undeclared variable \mintinline{c++}{ab} is used 
\begin{figure}[H]
  \inputminted{c++}{src/part1/ch3_ast/maxerr.cpp}
  \caption{The maxerr.cpp test program with a semantic error, undeclared
    variable} 
  \label{lis:min:maxerr}
\end{figure}

We are interested in the AST result produced by Clang, and we will use
\myshell{clang-query} to examine it, which can be run as follows:
% ./llvm-project-16.x/install/bin/clang-query ./src/part1/ch3_ast/maxerr.cpp
\begin{adjustwidth}{0em}{0em}
\begin{verbatim}
llvm-project/install/bin/clang-query maxerr.cpp
...
maxerr.cpp:3:12: error: use of undeclared identifier 'ab'
    return ab;
           ^
\end{verbatim}
\end{adjustwidth}
From the output, we can see that clang-query displayed a compilation error
detected by the compiler. It's worth noting that despite this, an AST was
produced for the program, and we can examine it. We are particularly interested
in the return statements and can use the corresponding matcher to highlight the
relevant parts of the AST.

We will also set up the output to produce the AST and search for return statements that are of interest to us
\begin{adjustwidth}{0em}{0em}
\begin{verbatim}
clang-query> set output dump
clang-query> match returnStmt()
\end{verbatim}
\end{adjustwidth}

The resulting output identifies two return statements in our program: the first
match on line 5 and the second match on line 3
\begin{figure}[H]
\begin{adjustwidth}{0em}{0em}
\begin{verbatim}
Match #1:

Binding for "root":
ReturnStmt 0x6b63230 <maxerr.cpp:5:3, col:10>
`-ImplicitCastExpr 0x6b63218 <col:10> 'int' <LValueToRValue>
  `-DeclRefExpr 0x6b631f8 <col:10> 'int' lvalue ParmVar 0x6b62ec8 'b' 'int'


Match #2:

Binding for "root":
ReturnStmt 0x6b631b0 <maxerr.cpp:3:5, col:12>
`-RecoveryExpr 0x6b63190 <col:12> '<dependent type>' contains-errors lvalue

2 matches.
\end{verbatim}
\end{adjustwidth}
\caption{ReturnStmt node matches at maxerr.cpp test program}
\label{lis:ch3:maxerr:returnmatch}
\end{figure}
As we can see, the first match corresponds to semantically correct code on line 5
and contains a reference to the parameter \mintinline{c++}{a}. The second match
is for line 3, which has a compilation error. Notably, Clang has inserted a
special type of AST node: \mintinline{c++}{RecoveryExpr}. It's worth noting that
in certain situations, Clang might produce an incomplete AST. This can cause
issues with Clang tools, such as lint checks. In instances of compilation
errors, lint checks might yield unexpected results because Clang couldn't
recover accurately from the compilation errors. We will revisit the problem when
exploring the clang-tidy lint check framework in \fullref{ch:clangtidy}. 
\label{sec:ast:errors}

\section{Summary}
We explored the Clang AST, a major instrument for creating various Clang
tools. We learned about the architectural design principles chosen for the
implementation of Clang AST and investigated different methods for AST
traversal. We delved into specialized traversal techniques, such as those for
C/C++ declarations, and also looked into more universal techniques that employ
recursive visitors and Clang AST matchers. Our exploration concluded with the
\myshell{clang-query} tool and how it can be used for Clang AST
exploration. Specifically, we used it to understand how Clang processes
compilation errors. 

The next chapter will discuss the basic libraries used in Clang and LLVM
development. We will explore the LLVM code style and foundational Clang/LLVM
classes, such as \mintinline{c++}{SourceManager} and
\mintinline{c++}{SourceLocation}. We will also cover the TableGen library, which
is used for code generation, and the LIT (LLVM Integration Test) framework. 

\section{Further reading}
\begin{itemize}
  \item How to write RecursiveASTVisitor:
    https://clang.llvm.org/docs/RAVFrontendAction.html
  \item AST Matcher Reference: https://clang.llvm.org/docs/LibASTMatchersReference.html
\end{itemize}


