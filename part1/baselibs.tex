%% -*- coding:utf-8 -*-
\chapter{Basic libraries and tools}
\pagestyle{fancy}
\fancyhf{}
\rhead{\thepage}
\lhead{Basic libraries and tools}
LLVM has been written in C++ language and currently (since July 2022) uses the
c++17 version of the C++ standard \citep{llvm:cpp17migration}. On the other side,
it has a lot of internal implementations for fundamental containers with the
primary goal of performance. Therefore, being familiar with the extensions is
crucial if you want to work with LLVM and clang. In addition, LLVM also
introduced additional development tools such as TableGen - DSL (domain-specific
language) for structural data processing and LIT (LLVM test framework). You can
find some info about the tools here. 

We are planning to use a simple example project to demonstrate the tools. The
example project will be \clang plugin that can estimate complexity of
the source code being compiled and print a warning if the number of
functions/methods exceeds the limit specified as a parameter.  

\section{LLVM coding style}
TBD

\section{Basic libs}
TBD

\section{TableGen}
TBD

\section{LLVM test framework}
\label{sec:LLVMTest}
TBD

\section{Summary}
TBD

\section{Further reading}
\begin{itemize}
\item LLVM Coding Standards: https://llvm.org/docs/CodingStandards.html \citep{llvm:coding_standards}
\end{itemize}

